%!TEX root=../main.tex
\section{Discussion and conclusion} % (fold)
\label{sec:discussion}

\add{MFPCA is a fundamental statistical tool utilized for the analysis of multivariate functional data, which enables to capture the variability in observations defined by multiple curves. In this paper, we have described the duality between rows and columns of a data matrix within the context of multivariate functional data. We have proposed to use this duality to estimate the eigencomponents of the covariance operator in multivariate functional datasets. By comparing the results of the two methods, we have provided the researcher with guidelines for determining the most appropriate method for application within a range of functional data frameworks. In summary, if the number of sampling points is significantly greater than the number of observations, or if the data at hand are multidimensional (e.g., surfaces), it is preferable to estimate the eigencomponents utilizing the Gram matrix. Conversely, if the data are unidimensional with a large number of observations, the preferred method is the direct decomposition of the covariance operator, regarless of the number of features.}

\add{Utilizing the Gram matrix enables the estimation of the number of components retained via the percentage of variance accounted for by the multivariate functional data, whereas the decomposition of the covariance operator necessitates the specification of the percentage of variance accounted for by each individual univariate feature. The necessity of specifying the percentage of variance explained for each features can not guarantees that we recover the right percentage of variance explained for the multivariate data.}

\add{In practice, observations of (multivariate) functional data are often subject to noise. As we recommend the use of the Gram matrix solely for densely sampled functional datasets, individual curve smoothing should suffice to approximate the Gram matrix in such cases. The estimation of the Gram matrix in the context of sparsely sampled functional data is however deemed irrelevant, given our findings that the utilization of the covariance operator for the estimation of the eigencomponents yields comparable results, while typically requiring less computational time.}

The open-source implementation can be accessed at \url{https://github.com/StevenGolovkine/FDApy}, while scripts to reproduce the simulation are at \url{https://github.com/FAST-ULxNUIG/geom_mfpca}.


% section discussion (end)