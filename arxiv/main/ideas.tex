%!TEX root=../main.tex
\section{Ideas} % (fold)
\label{sec:ideas}

\begin{itemize}
    \item Make the connection between \cite{happMultivariateFunctionalPrincipal2018a} and \cite{berrenderoPrincipalComponentsMultivariate2011}, \cite{yangFunctionalSingularComponent2011}?
    \item Concerning the estimation of functional principal components from the sample covariance matrix see \cite[Chap. 8.4]{ramsayFunctionalDataAnalysis2005} for univariate fPCA and \cite{happMultivariateFunctionalPrincipal2018a} for multivariate fPCA.
    \item Multiple correspondence analysis is equivalent to principal composant analysis of a transformed complete disjunctive table, see \cite{pagesMultipleFactorAnalysis2014}.
    \item How is this related to kernel PCA?
    \item Represent the components in a correlation circle where two components are close from one another if they exhibits the same variation in the space of principal components.
    \item Geometric of the different principal components.
    \item Comments on the geometric interpretation after the expansion in a common basis of functions such as B-splines or Fourier.
    \item ``[...] in multivariate analysis a substantial interpretation of principal components is often difficult and has to be based on vague arguments concerning the correlation of principal components with original variables. Such a problem does not at all exists in the functional context, where $\gamma_1(t), \gamma_2(t), \dots$ are functions representing the major modes of variation of $X_i(t)$ over $t$.'' \cite{benkoCommonFunctionalPrincipal2009}. It might however not be the case for multivariate functional data as the setting is more similar to the multivariate data.
\end{itemize}

% section ideas (end)