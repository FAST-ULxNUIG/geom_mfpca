%!TeX root=../main.tex
\section{Model} % (fold)
\label{sec:model}

The structure of the data we consider, referred to as \emph{multivariate functional data}, is very similar to that presented in \cite{happMultivariateFunctionalPrincipal2015}. The data consist of independent trajectories of a vector-valued stochastic process $X = (\Xp{1}, \dots, \Xp{P})^\top$, $P\geq 1$. (Here and in the following, for any matrix $A$, $A^\top$ denotes its transpose.) For each $1 \leq p \leq P$, let $\TT{p}$ be a rectangle in some Euclidean space $\RR^{d_p}$ with $d_p \geq 1$, \emph{e.g.}, $\TT{p} = [0,1]^{d_p}$. Each coordinate $X^{(p)} : \TT{p} \rightarrow \RR$ is assumed to belong to  $\sLp{\TT{p}}$, the Hilbert space of squared-integrable real-valued functions defined on $\TT{p}$, having the usual inner product that we denote by $\inLp{\cdot}{\cdot}$, and $\normLp{\cdot}$ the associated norm. Thus $X$ is a stochastic process indexed by $\pointt = (t_1, \ldots, t_P)$ belonging to the $P-$fold Cartesian product $\TT{} : =\TT{1} \times \cdots \times \TT{P}$ and taking values in the $P-$fold Cartesian product space $\HH \coloneqq \sLp{\TT{1}} \times \dots \times \sLp{\TT{P}}$. 

We consider the function $\inH{\cdot}{\cdot} : \HH \times \HH \rightarrow \RR$,
\begin{equation}\label{eq:innerprodH}
    \inH{f}{g} \coloneqq \sum_{p=1}^{P} \inLp{\fp}{\gp} = \sum_{p=1}^{P}\int_{\TT{p}} \fp(t_p)\gp(t_p) \dd t_p, \quad f, g \in \HH.
\end{equation}
$\HH$ is a Hilbert space with respect to the inner product $\inH{\cdot}{\cdot}$\citep{happMultivariateFunctionalPrincipal2015}. We denote by $\normH{\cdot}$, the norm induced by $\inH{\cdot}{\cdot}$. Let $\mu : \TT{} \rightarrow \HH$ denote the mean function of the process $X$, $\mu(\pointt) \coloneqq \EE(X(\pointt)), \pointt \in \TT{}$. Let $C$ denotes the $P \times P$ matrix-valued covariance function which, for $\points, \pointt \in \TT{}$, is defined as
\begin{equation}\label{eq:covariance_function}
    C(\points, \pointt) \coloneqq \EE\left(\{X(\points) - \mu(\points)\}\{X(\pointt) - \mu(\pointt)\}^{\top}\right), \quad \points, \pointt \in \TT{}.
\end{equation}
More precisely, for $1 \leq p, q \leq P$, the $(p, q)$th entry of the matrix $C(\points, \pointt)$ is the covariance function between the $p$th and the $q$th components of the process $X$:
\begin{equation}\label{eq:covariance_function_components}
    C_{p, q}(s_p, t_q) \coloneqq \EE\left(\{\Xp{p}(s_p) - \mup{p}(s_p)\}\{\Xp{q}(t_q) - \mup{q}(t_q)\}\right), \quad s_p \in \TT{p}, t_q \in \TT{q}.
\end{equation}
Let $\Gamma : \HH \rightarrow \HH$ denotes the covariance operator of $X$, defined as an integral operator with kernel $C$. That is, for $f \in \HH$ and $\pointt \in \TT{}$, the $p$th component of $\Gamma f(\pointt)$ is given by
\begin{equation}\label{eq:covariance_operator_components}
    (\Gamma f)^{(p)}(t_p) \coloneqq \inH{C_{p, \cdot}(t_p, \cdot)}{f(\cdot)} = \inH{C_{\cdot, p}(\cdot, t_p)}{f(\cdot)}, \quad t_p \in \TT{p}.
\end{equation}

Let us consider the set of $N$ curves $\XX = \{X_1, \ldots, X_n, \ldots, X_N\}$ generated as a random sample of the $P$-dimensional stochastic process $X$ with continuous trajectories. Unless otherwise stated, the data are assumed to be observed without error. The data can be viewed as a table with $N$ rows and $P$ columns where each entry is a curve, eventually multidimensional (see Figure~\ref{fig:data_matrix}). Each row of this matrix represents an observation; while each column represents a functional variable. At the intersection of row $n$ and column $p$, we thus have $\Xnp$ which is the curve that concerns the feature $p$ for the individual $n$.

\begin{figure}
    \centering
    \includegraphics[]{figures/data_matrix.pdf}
    \caption{Data matrix, adapted from \cite{berrenderoPrincipalComponentsMultivariate2011}.}
    \label{fig:data_matrix}
\end{figure}

For $n \in \{1, \dots, N\}$, each observation $n$ is attributed the weight $\pi_n$ such that $\sum_n \pi_n = 1$, e.g., $\pi_n = 1/N$.
For a given $p \in \{1, \dots, P\}$, the mean curve of the $p$th component along the $N$ observations is denoted by $\mup{p}$. This quantity can be computed using 
\begin{equation*}\label{eq:mu_estimation}
    \mup{p}(t_p) = \sum_{n = 1}^N \pi_n\Xnp(t_p), \quad t_p \in \TT{p}, \quad p \in \{1, \dots, P\}.
\end{equation*}
The covariance function of the $p$th component along the $N$ observations can be computed using
\begin{equation}\label{eq:cov_estimation}
    C_{p, p}(s_p, t_p) = \sum_{n = 1}^N \pi_n\Xnp(s_p)\Xnp(t_p) - \mup{p}(s_p)\mup{p}(t_p), \quad s_p, t_p \in \TT{p}, \quad p \in \{1, \dots, P\}.
\end{equation}

% section model (end)