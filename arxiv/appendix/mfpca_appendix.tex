%!TEX root=../main.tex
\section{Derivation of the eigencomponents} % (fold)
\label{sec:derivation_of_the_eigencomponents}

\subsection{General framework} % (fold)
\label{sub:general_framework}
In this section, we calculate the relationships between the eigenelements of the covariance operator $\Gamma$ and the ones of the Gram matrix $\mathbf{M}$ of a functional dataset. We then prove the equalities~\eqref{eq:eigenvalues_relation_p},~\eqref{eq:eigenfunction_relation_p} and~\eqref{eq:scores_relation_p}.

Using the Hilbert-Schmidt theorem, there exists a complete orthonormal basis of eigenvectors $\{\boldsymbol{u}_k\}_{1 \leq k \leq N}$ of the inner-product matrix $\mathbf{M}$ such that
\begin{equation}\label{eq:eigen_inner_prod_p}
    \mathbf{M}\boldsymbol{u}_k = l_k\boldsymbol{u}_k.
\end{equation}
Let $X = \left(X_1 - \mu, \dots, X_N - \mu\right)^\top$ and denote $\widetilde{X} = \text{diag}\{\sqrt{\pi_1}, \dots, \sqrt{\pi_N}\}X$, the matrix of weighted observations. Recall that, in the case of $P$-dimensional process, the realisations of the process $X_n,~n = 1, \cdots, N$ and $\mu$ are vectors of functions of length $P$, and thus $X$ (and $\widetilde{X}$) is a matrix of functions of size $N \times P$. By left multiplying Equation~\eqref{eq:eigen_inner_prod_p} by $\widetilde{X}^\top$, we obtain
\begin{equation}\label{eq:eigen_inner_prod_left}
    \widetilde{X}^\top \mathbf{M} \boldsymbol{u}_k = l_k \widetilde{X}^\top \boldsymbol{u}_k.
\end{equation} 
Expanding Equation~\eqref{eq:eigen_inner_prod_left}, for each component $p = 1, \dots, P$, we have,
\begin{equation}\label{eq:inner_prod_p}
    \sum_{i = 1}^N \sum_{j = 1}^N \pi_i \sqrt{\pi_j}[\boldsymbol{u}_{k}]_j\mkern-4mu\left\{X_i^{(p)}(\cdot) - \mu^{(p)}(\cdot)\right\}\inH{X_i - \mu}{X_j - \mu} = l_k \mkern-5mu\sum_{n = 1}^N \mkern-4mu\sqrt{\pi_n}[\boldsymbol{u}_{k}]_n\mkern-4mu\left\{\Xnp(\cdot) - \mu^{(p)}(\cdot)\right\}.
\end{equation}
Here and in the following, we note $[a]_p$ the $p$th entry of the vector $a$. Starting from the left side of Equation~\eqref{eq:inner_prod_p}, we get
\begin{align}\label{eq:inner_prod_p_left}
[\widetilde{X}^\top \mathbf{M} \boldsymbol{u}_k]_p &= \sum_{i = 1}^N \sum_{j = 1}^N \pi_i \sqrt{\pi_j} [\boldsymbol{u}_{k}]_j \left\{X_i^{(p)}(\cdot) - \mu^{(p)}(\cdot)\right\}\inH{X_i - \mu}{X_j - \mu}\\
&= \sum_{q = 1}^P \int_{\TT{q}} \sum_{i = 1}^N \pi_i\left\{X_i^{(p)}(\cdot) - \mu^{(p)}(\cdot)\right\} \left\{X_i^{(q)}(s_q) - \mu^{(q)}(s_q)\right\}  \\
&\qquad\qquad \times \sum_{j = 1}^N \sqrt{\pi_j}[\boldsymbol{u}_{k}]_j \left\{X_j^{(q)}(s_q) - \mu^{(q)}(s_q)\right\} \dd s_q \\
&= \sum_{q = 1}^P \int_{\TT{q}} C_{p,q}(\cdot, s_q)\sum_{j = 1}^N \sqrt{\pi_j}[\boldsymbol{u}_{k}]_j \left\{X_j^{(q)}(s_q) - \mu^{(q)}(s_q)\right\} \dd s_q \\
&= \sum_{j = 1}^N \inH{C_{p \cdot}(\cdot, \cdot)}{\sqrt{\pi_j}[\boldsymbol{u}_{k}]_j \left\{X_j - \mu\right\}} \\
&= \Gamma\left(\sum_{j = 1}^N \sqrt{\pi_j}[\boldsymbol{u}_{k}]_j \left\{X_j - \mu\right\} \right)^{\mkern-9mu(p)}\mkern-18mu(\cdot),
\end{align}
and, starting from the right side of Equation~\eqref{eq:inner_prod_p},
\begin{equation}\label{eq:inner_prod_p_right}
    [l_k \widetilde{X}^\top \boldsymbol{u}_k]_p = l_k \sum_{n = 1}^N \sqrt{\pi_n}[\boldsymbol{u}_{k}]_n \left\{\Xnp(\cdot) - \mu^{(p)}(\cdot)\right\}.
\end{equation}
From Equation~\eqref{eq:inner_prod_p_left} and Equation~\eqref{eq:inner_prod_p_right}, we obtain
\begin{equation}
    \Gamma\left(\sum_{j = 1}^N \sqrt{\pi_j}[\boldsymbol{u}_{k}]_j \left\{X_j - \mu\right\}\right)^{\mkern-9mu(p)}\mkern-18mu(\cdot) = l_k \sum_{n = 1}^N \sqrt{\pi_n}[\boldsymbol{u}_{k}]_n \left\{\Xnp(\cdot) - \mu^{(p)}(\cdot)\right\}, \quad\text{for all}~ p = 1, \dots, P.
\end{equation}
By identification in Equation~\eqref{eq:eigendecomposition}, we find that, for each components $p$,
\begin{equation}\label{eq:eigen_estimation}
\lambda_k = l_k \quad\text{and}\quad \phi_k^{(p)}(\cdot) = \sum_{n = 1}^N \sqrt{\pi_n}[\boldsymbol{u}_{k}]_n \left\{\Xnp(\cdot) - \mu^{(p)}(\cdot)\right\}, \quad k \geq 1.
\end{equation}
For $k \geq 1$, the norm of the eigenfunction is computed as the following:
\begin{align*}
\normH{\phi_k}^2 &= \sum_{i = 1}^N \sum_{j = 1}^N \sqrt{\pi_i\pi_j}[\boldsymbol{u}_{k}]_i [\boldsymbol{u}_{k}]_j\inH{X_i - \mu}{X_j - \mu} = \sum_{i = 1}^N [\boldsymbol{u}_{k}]_i \sum_{j = 1}^N \mathbf{M}_{ij} [\boldsymbol{u}_{k}]_j \\
    &= \sum_{i = 1}^N [\boldsymbol{u}_{k}]_i l_k [\boldsymbol{u}_{k}]_i = l_k \normR{\boldsymbol{u}_k}^2 = l_k. \\
\end{align*}
Therefore, in order to have an orthonormal basis of eigenfunctions, we normalise the eigenfunctions $\phi_k$ from Equation~\eqref{eq:eigen_estimation} by $1 / \sqrt{l_k}$.
Concerning the estimation of the scores, for $n = 1, \dots, N$, for $k \geq 1$, we have
\begin{align}
    \mathfrak{c}_{nk} &= \inH{X_n - \mu}{\phi_k} = \frac{1}{\sqrt{l_k}}\sum_{j = 1}^N \sqrt{\pi_j}[\boldsymbol{u}_{k}]_j \inH{X_n - \mu}{X_j - \mu}\\
    &= \frac{1}{\sqrt{l_k\pi_n}}\sum_{j = 1}^N [\boldsymbol{u}_{k}]_j \mathbf{M}_{nj} = \sqrt{\frac{l_k}{\pi_n}}[\boldsymbol{u}_{k}]_n.\\
\end{align}

If we assume that the observations are equally weighted, i.e., $\pi_n = 1 / N, n = 1, \dots, N$, we get the equalities~\eqref{eq:eigenvalues_relation_p},~\eqref{eq:eigenfunction_relation_p} and~\eqref{eq:scores_relation_p}.

% subsection general_framework (end)

\subsection{With a basis expansion} % (fold)
\label{sub:with_a_basis_expansion}

In this section, we assume that the observations are expanded into a basis of functions, as explained in Section~\ref{sub:basis_decomposition}. Using the expansion of the data into the basis of function $\Psi$ and $\mathbf{W}$, the matrix of inner products of the functions in the basis $\Psi$, we write the Equation~\eqref{eq:gram_mat} as
\begin{equation}
    \mathbf{M} = \left(\text{diag}\{
        \sqrt{\pi_1}, \dots, \sqrt{\pi_N}\}\left(\mathrm{I}_{\!N} - \mathbf{1}_{\!N}\Pi^\top\right) \mathbf{C}\mathbf{W}^{1/2}\right)\left(\text{diag}\{
        \sqrt{\pi_1}, \dots, \sqrt{\pi_N}\}\left(\mathrm{I}_{\!N} - \mathbf{1}_{\!N}\Pi^\top\right) \mathbf{C}\mathbf{W}^{1/2}\right)^\top.
\end{equation}
We note
\begin{equation}
    \mathbf{A} = \text{diag}\{\sqrt{\pi_1}, \dots, \sqrt{\pi_N}\}\left(\mathrm{I}_{\!N} - \mathbf{1}_{\!N}\Pi^\top\right) \mathbf{C}\mathbf{W}^{1/2},
\end{equation}
such that $\mathbf{M} = \mathbf{A}\mathbf{A}^\top$.
We also assume that $\phi_1, \phi_2, \dots$ the eigenfunctions of the covariance operator $\Gamma$ have a decomposition into the basis $\Psi$
\begin{equation}
    \phi_k(\cdot) = 
        \begin{pmatrix} 
            \phi_k^{(1)}(\cdot) \\
            \vdots \\
            \phi_k^{(P)}(\cdot)
        \end{pmatrix} = 
        \begin{pmatrix} 
            \psi^{(1) \top}(\cdot) b_{1k} \\
            \vdots \\
            \psi^{(P) \top}(\cdot) b_{Pk}
        \end{pmatrix}, \quad\text{where}\quad
        b_{pk} = \left(b_{p k 1}, \dots, b_{p k K_p} \right)^\top.
\end{equation}
We have, for $p = 1, \dots, P$,
\begin{align*}
    \left(\Gamma \phi_k\right)^{(p)}(\cdot) &= \sum_{q = 1}^P \int_{\TT{q}} C_{p, q}(\cdot, s_q)\phi_k^{(q)}(s_q) \dd s_q \\
    &= \sum_{q = 1}^P \int_{\TT{q}} \Psi(\cdot)^{(p) \top} \mathbf{C}^{(p) \top} \left(\text{diag}\{\pi_1, \dots, \pi_N\} - \Pi\Pi^\top\right)\mathbf{C}^{(q)} \Psi^{(q)}(s_q) \Psi^{(q)}(s_q)^\top b_{q k} \dd s_q \\
    &= \Psi(\cdot)^{(p) \top} \mathbf{C}^{(p) \top} \left(\text{diag}\{\pi_1, \dots, \pi_N\} - \Pi\Pi^\top\right)\sum_{q = 1}^P \mathbf{C}^{(q)} \int_{\TT{q}} \Psi^{(q)}(s_q) \Psi(s_q)^{(q) \top} \dd s_q b_{q k} \\
    &= \Psi(\cdot)^{(p) \top} \mathbf{C}^{(p) \top} \left(\text{diag}\{\pi_1, \dots, \pi_N\} - \Pi\Pi^\top\right) \sum_{q = 1}^P \mathbf{C}^{(q)} \mathbf{W}^{(q)} b_{q k}. \\
\end{align*}
This equation is true for all $p = 1, \cdots, P$, this can be rewritten with matrices as
\begin{equation}
    \Gamma \phi_k(\cdot) = \Psi(\cdot)^{\top} \mathbf{C}^{\top} \left(\text{diag}\{\pi_1, \dots, \pi_N\} - \Pi\Pi^\top\right) \mathbf{C} \mathbf{W} b_{k}.
\end{equation}
From the eigenequation, we have that
\begin{equation}
    \Gamma \phi_k(\cdot) = \lambda_k \phi_k(\cdot) \Longleftrightarrow \Psi(\cdot)^{\top} \mathbf{C}^{\top} \left(\text{diag}\{\pi_1, \dots, \pi_N\} - \Pi\Pi^\top\right) \mathbf{C} \mathbf{W} b_{k} = \lambda_k \Psi(\cdot)^\top b_k.
\end{equation}
Since this equation must be true for all $t_p \in \TT{p}$, this imply the equation
\begin{equation}\label{eq:eigen_decom}
    \mathbf{C}^{\top} \left(\text{diag}\{\pi_1, \dots, \pi_N\} - \Pi\Pi^\top\right) \mathbf{C} \mathbf{W} b_{k} = \lambda_k b_k.
\end{equation}
As the eigenfunctions are assumed to be normalized, $\normH{\phi_k}^2 = 1$. And so, $b_k^\top \mathbf{W} b_k = 1$. Let $u_k = \mathbf{W}^{1/2}b_k$. Then, from Equation \eqref{eq:eigen_decom}, we obtain
\begin{equation}\label{eq:eigen_cov_op}
    \mathbf{W}^{1/2} \mathbf{C}^{\top} \left(\text{diag}\{\pi_1, \dots, \pi_N\} - \Pi\Pi^\top\right) \mathbf{C} \mathbf{W}^{1/2} u_k = \lambda_k u_k \Longleftrightarrow \mathbf{A}^\top\mathbf{A} u_k = \lambda_k u_k.
\end{equation}
From the eigendecomposition of the matrix $M$, we get
\begin{equation}\label{eq:eigen_inner_prod}
    \mathbf{M}\boldsymbol{u}_k = l_k \boldsymbol{u}_k \Longleftrightarrow \mathbf{A}\mathbf{A}^\top \boldsymbol{u}_k = l_k \boldsymbol{u}_k.
\end{equation}
The equations \eqref{eq:eigen_cov_op} and \eqref{eq:eigen_inner_prod} are eigenequations in the classical PCA case, with the duality $X^\top X$ and $XX^\top$. Following \cite{pagesMultipleFactorAnalysis2014,hardleAppliedMultivariateStatistical2019}, we find that, for $1 \leq k \leq K$,
\begin{equation}
    \lambda_k = l_k, \quad \boldsymbol{u}_k = \frac{1}{\sqrt{l_k}}\mathbf{A} u_k \quad\text{and}\quad u_k = \frac{1}{\sqrt{l_k}} \mathbf{A}^\top \boldsymbol{u}_k.
\end{equation}
And finally, to get the coefficient of the eigenfunctions, for $1 \leq k \leq K$,
\begin{equation}
    b_k = \mathbf{W}^{-1/2}u_k = \frac{1}{\sqrt{l_k}} \mathbf{C}^\top \left(\mathrm{I}_{\!N} - \Pi\mathbf{1}_{\!N}^\top\right) \text{diag}\{\sqrt{\pi_1}, \dots, \sqrt{\pi_N}\}\boldsymbol{u}_k.
\end{equation}

% subsection with_a_basis_expansion (end)

% section derivation_of_the_eigencomponents (end)