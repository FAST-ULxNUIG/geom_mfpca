%!TEX root=../main.tex
\section{Derivation of the eigencomponents} % (fold)
\label{sec:derivation_of_the_eigencomponents}

Using the Hilbert-Schmidt theorem, there exists a complete orthonormal basis of eigenvectors $\{v_k\}_{1 \leq k \leq N}$ of the inner-product matrix $\mathbf{M}$ such that
\begin{equation}\label{eq:eigen_inner_prod_p}
    \mathbf{M}v_k = l_kv_k.
\end{equation}
Let $X = \left(X_1 - \mu, \dots, X_N - \mu\right)^\top$. Recall that, in the case of $P$-dimensional process, the realisations of the process $X_n,~n = 1, \cdots, N$ and $\mu$ are vectors of functions of length $P$, and thus $X$ is a matrix of functions of size $N \times P$. By left multiplying Equation~\eqref{eq:eigen_inner_prod_p} by $X^\top$, we obtain
\begin{equation}\label{eq:eigen_inner_prod_left}
    X^\top \mathbf{M} v_k = l_k X^\top v_k.
\end{equation} 
Expanding Equation~\eqref{eq:eigen_inner_prod_left}, for each component $p = 1, \dots, P$, we have,
\begin{equation}\label{eq:inner_prod_p}
    \sum_{i = 1}^N \sum_{j = 1}^N [v_{k}]_j\left\{X_i^{(p)}(\cdot) - \mu^{(p)}(\cdot)\right\}\inH{X_i - \mu}{X_j - \mu} = l_k \sum_{n = 1}^N [v_{k}]_n\left\{\Xnp(\cdot) - \mu^{(p)}(\cdot)\right\}.
\end{equation}
Here and in the following, we note $[a]_p$ the $p$th entry of the vector $a$. Starting from the left side of Equation~\eqref{eq:inner_prod_p}, we get
\begin{align}\label{eq:inner_prod_p_left}
[X^\top \mathbf{M} v_k]_p &= \sum_{i = 1}^N \sum_{j = 1}^N [v_{k}]_j \left\{X_i^{(p)}(\cdot) - \mu^{(p)}(\cdot)\right\}\inH{X_i - \mu}{X_j - \mu}\\
&= \sum_{q = 1}^P \int_{\TT{q}} \sum_{i = 1}^N \left\{X_i^{(p)}(\cdot) - \mu^{(p)}(\cdot)\right\} \left\{X_i^{(q)}(s_q) - \mu^{(q)}(s_q)\right\}  \\
&\quad\quad \sum_{j = 1}^N [v_{k}]_j \left\{X_j^{(q)}(s_q) - \mu^{(q)}(s_q)\right\} \dd s_q \\
&= \sum_{q = 1}^P \int_{\TT{q}} N C_{pq}(\cdot, s_q)\sum_{j = 1}^N [v_{k}]_j \left\{X_j^{(q)}(s_q) - \mu^{(q)}(s_q)\right\} \dd s_q \\
&= N \sum_{j = 1}^N \inH{C_{p \cdot}(\cdot, \cdot)}{[v_{k}]_j \left\{X_j - \mu\right\}} \\
&= N \Gamma\left(\sum_{j = 1}^N [v_{k}]_j \left\{X_j - \mu\right\} \right)^{\mkern-9mu(p)}\mkern-18mu(\cdot)
\end{align}
and, starting from the right side of Equation~\eqref{eq:inner_prod_p},
\begin{equation}\label{eq:inner_prod_p_right}
    [l_k X^\top v_k]_p = l_k \sum_{n = 1}^N [v_{k}]_n \left\{\Xnp(\cdot) - \mu^{(p)}(\cdot)\right\}.
\end{equation}
From Equation~\eqref{eq:inner_prod_p_left} and Equation~\eqref{eq:inner_prod_p_right}, we obtain
\begin{equation}
    \Gamma\left(\sum_{j = 1}^N [v_{k}]_j \left\{X_j - \mu\right\}\right)^{\mkern-9mu(p)}\mkern-18mu(\cdot) = \frac{l_k}{N} \sum_{n = 1}^N [v_{k}]_n \left\{\Xnp(\cdot) - \mu^{(p)}(\cdot)\right\}, \quad\text{for all}~ p = 1, \dots, P.
\end{equation}
By identification in Equation~\eqref{eq:eigendecomposition}, we find that, for each components $p$,
\begin{equation}\label{eq:eigen_estimation}
\lambda_k = \frac{l_k}{N} \quad\text{and}\quad \phi_k^{(p)}(\cdot) = \sum_{n = 1}^N [v_{k}]_n \left\{\Xnp(\cdot) - \mu^{(p)}(\cdot)\right\}, \quad k \geq 1.
\end{equation}
For $k \geq 1$, the norm of the eigenfunction is computed as the following:
\begin{align*}
\normH{\phi_k}^2 &= \sum_{i = 1}^N \sum_{j = 1}^N [v_{k}]_i [v_{k}]_j\inH{X_i - \mu}{X_j - \mu} = \sum_{i = 1}^N [v_{k}]_i \sum_{j = 1}^N \mathbf{M}_{ij} [v_{k}]_j \\
    &= \sum_{i = 1}^N [v_{k}]_i l_k [v_{k}]_i = l_k \normLp{v_k}^2 = l_k. \\
\end{align*}
Therefore, in order to have an orthonormal basis of eigenfunctions, we normalise the eigenfunctions $\phi_k$ from Equation~\eqref{eq:eigen_estimation} by $1 / \sqrt{l_k}$.
Concerning the estimation of the scores, for $n = 1, \dots, N$, for $k \geq 1$, we have
\begin{align}
    \widehat{\mathfrak{c}}_{nk} &= \inH{X_n - \mu}{\phi_k} = \frac{1}{\sqrt{l_k}}\sum_{j = 1}^N [v_{k}]_j \inH{X_n - \mu}{X_j - \mu}\\
    &= \frac{1}{\sqrt{l_k}}\sum_{j = 1}^N [v_{k}]_j \mathbf{M}_{nj} = \sqrt{l_k}[v_{k}]_n.\\
\end{align}
\textcolor{red}{Rewrite this with centering.}
Concerning the expansion of the data into the basis of function $\Psi$, we write 
\begin{equation}
    M = \left(CW^{1/2}\right)\left(CW^{1/2}\right)^\top.
\end{equation}
We also assume that $\widehat{\phi}_1, \widehat{\phi}_2, \dots$ the eigenfunctions of $\widehat{\Gamma}_N$ have a decomposition into the basis $\Psi$
\begin{equation}
    \widehat{\phi}_k(\cdot) = 
        \begin{pmatrix} 
            \widehat{\phi}_k^{(1)}(\cdot) \\
            \vdots \\
            \widehat{\phi}_k^{(P)}(\cdot)
        \end{pmatrix} = 
        \begin{pmatrix} 
            \psi^{(1) \top}(\cdot) b_{1k} \\
            \vdots \\
            \psi^{(P) \top}(\cdot) b_{Pk}
        \end{pmatrix}, \quad\text{where}\quad
        b_{pk} = \left(b_{p k 1}, \dots, b_{p k K_p} \right)^\top.
\end{equation}

We have, for $p = 1, \dots, P$,
\begin{align*}
    \left(\widehat{\Gamma}_N \phi_k\right)^{(p)}(\cdot) &= \sum_{q = 1}^P \int_{\TT{q}} \widehat{C}_{p q}(\cdot, s_q)\phi_k^{(q)}(s_q) \dd s_q \\
    &= \frac{1}{N}\sum_{q = 1}^P \int_{\TT{q}} \Psi(\cdot)^{(p) \top} C^{(p) \top} C^{(q)} \Psi^{(q)}(s_q) \Psi^{(q)}(s_q)^\top b_{q k} \dd s_q \\
    &= \frac{1}{N} \Psi(\cdot)^{(p) \top} C^{(p) \top} \sum_{q = 1}^P C^{(q)} \int_{\TT{q}} \Psi^{(q)}(s_q) \Psi(s_q)^{(q) \top} \dd s_q b_{q k} \\
    &= \frac{1}{N} \Psi(\cdot)^{(p) \top} C^{(p) \top} \sum_{q = 1}^P C^{(q)} W^{(q)} b_{q k}. \\
\end{align*}
This equation is true for all $p = 1, \cdots, P$, this can be rewritten with matrices as
\begin{equation}
    \widehat{\Gamma}_N \phi_k(\cdot) = \frac{1}{N}\Psi^\top(\cdot) C^\top C W b_k.
\end{equation}
From the eigenequation, we have that
\begin{equation}
    \widehat{\Gamma}_N \phi_k(\cdot) = \lambda_k \phi_k(\cdot) \Longleftrightarrow \frac{1}{N} \Psi(\cdot)^\top C^\top C W b_k = \lambda_k \Psi(\cdot)^\top b_k.
\end{equation}
Since this equation must be true for all $t_p \in \TT{p}$, this imply the equation
\begin{equation}\label{eq:eigen_decom}
    C^\top C W b_k = N\lambda_k b_k.
\end{equation}
As the eigenfunctions are assumed to be normalized, $\normH{\phi_k}^2 = 1$. And so, $b_k^\top W b_k = 1$. Let $u_k = W^{1/2}b_k$. Then, from Equation \eqref{eq:eigen_decom}, we obtain
\begin{equation}\label{eq:eigen_cov_op}
    W^{1/2} C^\top C W^{1/2} u_k = N\lambda_k u_k \Longleftrightarrow \left(C W^{1/2}\right)^\top \left(C W^{1/2}\right) u_k = N\lambda_k u_k.
\end{equation}
From the eigendecomposition of the matrix $M$, we get
\begin{equation}\label{eq:eigen_inner_prod}
    Mv_k = l_k v_k \Longleftrightarrow \left(C W^{1/2}\right)\left(C W^{1/2}\right)^\top v_k = l_k v_k.
\end{equation}
The equations \eqref{eq:eigen_cov_op} and \eqref{eq:eigen_inner_prod} are eigenequations in the classical PCA case, with the duality $X^\top X$ and $XX^\top$. Following \cite{pagesMultipleFactorAnalysis2014,hardleAppliedMultivariateStatistical2019}, we find that, for $1 \leq k \leq K$,
\begin{equation}
    \lambda_k = \frac{l_k}{N}, \quad v_k = \frac{1}{\sqrt{l_k}}C W^{1/2} u_k \quad\text{and}\quad u_k = \frac{1}{\sqrt{l_k}} W^{1/2} C^\top v_k.
\end{equation}
And finally, to get the coefficient of the eigenfunctions, for $1 \leq k \leq K$,
\begin{equation}
    b_k = W^{-1/2}u_k = \frac{1}{\sqrt{l_k}} C^\top v_k.
\end{equation}

% section derivation_of_the_eigencomponents (end)